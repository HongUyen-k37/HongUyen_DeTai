\usepackage[utf8]{vietnam}

%----------------------------------------------------------
%Thiết lập lề trang giấy
\usepackage{changepage}
%\usepackage{anysize}
%\marginsize{3.3cm}{2.2cm}{1.3cm}{2.4cm}{geometry}
%\usepackage{blindtext}
\usepackage[a4paper,bindingoffset=1.0cm,left=2.3cm,right=2.3cm,top=2.5cm,bottom=3.0cm,footskip=1.2cm]{geometry}

%----------------------------------------------------------
%Sử dụng các goi
\usepackage{latexsym,amssymb,amsmath,mathrsfs,amsfonts,amsthm}
\usepackage{array}
\usepackage{tocloft}
\usepackage{etex}
\usepackage{url}
\usepackage{multirow}
\usepackage[all]{xy}
\usepackage{xspace}
\usepackage{stmaryrd}
\usepackage[ruled,vlined,linesnumbered,resetcount,algochapter]{algorithm2e}
\usepackage{algorithmic}
%Các gói dùng cho hình ảnh
\usepackage{graphicx}
\usepackage{tikz}
\usetikzlibrary{calc}
\usetikzlibrary{shapes}
\usepackage[caption=false,font=footnotesize]{subfig}
%----------------------------------------------------------
\usepackage{xcolor}
\usepackage{color}
\usepackage{titlesec}
\usepackage{hyperref}
\usepackage{enumitem}
%----------------------------------------------------------
%Dùng cho bảng biểu
\usepackage{tabularx}
\usepackage{longtable}
\usepackage{slashbox}
%Lùi đầu dòng cho đoạn đầu tiên của một mục
\usepackage{indentfirst}
%----------------------------------------------------------
%Thiết lập khoảng cách giữa các dòng và các đoạn
\raggedbottom
\setlength{\parindent}{20pt}
\setlength{\parskip}{5pt}
\renewcommand{\baselinestretch}{1.3}
%Thiết lập khoảng cách giữa các item
\setlist[itemize]{parsep=3pt}
\setlist[enumerate]{parsep=3pt}

%----------------------------------------------------------
%Dung cho Header và footer
\usepackage{fancyhdr}
	\pagestyle{fancy}
	\fancyhead[C]{}
	\fancyhead[RE,LO]{}
	\fancyhead[LE,RO]{\textsc{đề tài nghiên cứu khoa học cấp cơ sở đại học huế}}
	\fancyfoot[C]{}
	\fancyfoot[RE,LO]{}
	\fancyfoot[LE,RO]{\thepage}
	\renewcommand{\headrulewidth}{1pt}
	\renewcommand{\footrulewidth}{1pt}
	
%Định nghĩa các các ký hiệu
%-------------------------------------------------------------
\newcommand{\mL}		{\mathcal{L}}
\newcommand{\mG}		{\mathcal{G}}
\newcommand{\mA}		{\mathcal{A}}
\newcommand{\mT}		{\mathcal{T}}
\newcommand{\mR}		{\mathcal{R}}
\newcommand{\mI}		{\mathcal{I}}
\newcommand{\mC}		{\mathcal{C}}
\newcommand{\mE}		{\mathcal{E}}
\newcommand{\mP}		{\mathcal{P}}
\newcommand{\mS}		{\mathcal{S}}
\newcommand{\mH}		{\mathcal{H}}
\newcommand{\mO}		{\mathcal{O}}
\newcommand{\mN}		{\mathcal{N}}
\newcommand{\mQ}		{\mathcal{Q}}
\newcommand{\mF}		{\mathcal{F}}
\newcommand{\mD}		{\mathcal{D}}
\newcommand{\mU}		{\mathcal{U}}

\newcommand{\mbC}		{\mathbb{C}}
\newcommand{\mbD}		{\mathbb{D}}
\newcommand{\mbY}		{\mathbb{Y}}
\newcommand{\mbJ}		{\mathbb{J}}

\newcommand{\SigmaI}		{\Sigma_I}
\newcommand{\SigmaA}		{\Sigma_A}
\newcommand{\SigmaC}		{\Sigma_C}
\newcommand{\SigmaR}		{\Sigma_R}
\newcommand{\SigmaDA}		{\Sigma_{dA}}
\newcommand{\SigmaNA}		{\Sigma_{nA}}
\newcommand{\SigmaOR}		{\Sigma_{oR}}
\newcommand{\SigmaDR}		{\Sigma_{dR}}

\newcommand{\SigmaDag}		{\Sigma^\dag}
\newcommand{\SigmaDagI}		{\Sigma^\dag_I}
\newcommand{\SigmaDagA}		{\Sigma^\dag_A}
\newcommand{\SigmaDagC}		{\Sigma^\dag_C}
\newcommand{\SigmaDagR}		{\Sigma^\dag_R}
\newcommand{\SigmaDagDA}	{\Sigma^\dag_{dA}}
\newcommand{\SigmaDagNA}	{\Sigma^\dag_{nA}}
\newcommand{\SigmaDagOR}	{\Sigma^\dag_{oR}}
\newcommand{\SigmaDagDR}	{\Sigma^\dag_{dR}}
\newcommand{\PhiDag}		{\Phi^\dag}

\newcommand{\Attrs}			{\mathit{Attrs}}
\newcommand{\True}			{\mathsf{true}}
\newcommand{\False}			{\mathsf{false}}
\newcommand{\Self}			{\mathsf{Self}}
\newcommand{\KB}			{\mathcal{KB}}
\newcommand{\mLS}			{\mL_\Sigma}
\newcommand{\mLSD}			{\mL_{\Sigma^\dag}}
\newcommand{\mLSP}			{\mL_{\Sigma,\Phi}}
\newcommand{\mLSPD}			{\mL_{\Sigma^\dag,\Phi^\dag}}
\newcommand{\SdI}			{{\SigmaDag,\mI}}
\newcommand{\SdPdI}			{{\SigmaDag,\Phi^\dag,\mI}}
\newcommand{\simSdI}		{\sim_{\SigmaDag,\mI}}
\newcommand{\simSdPdI}		{\sim_{\SigmaDag,\Phi^\dag,\mI}}
\newcommand{\LargestContainer}{\mathit{LargestContainer}}

%--------------------------------------------------------------------
\newcommand{\FL}		{$\mathcal{FL}$\xspace}
\newcommand{\FLzero}	{$\mathcal{FL}_0$\xspace}
\newcommand{\FLbot}		{$\mathcal{FL}_\bot$\xspace}
\newcommand{\AL}		{$\mathcal{AL}$\xspace}
\newcommand{\ALC}		{$\mathcal{ALC}$\xspace}
\newcommand{\ALCreg}	{$\mathcal{ALC}_{reg}$\xspace}
\newcommand{\ALN}		{$\mathcal{ALN}$\xspace}
\newcommand{\ALCI}		{$\mathcal{ALCI}$\xspace}
\newcommand{\ALCN}		{$\mathcal{ALCN}$\xspace}
\newcommand{\ALCQ}		{$\mathcal{ALCQ}$\xspace}
\newcommand{\ALCIQ}		{$\mathcal{ALCIQ}$\xspace}
\newcommand{\ALER}		{$\mathcal{ALER}$\xspace}
\newcommand{\LogicS}	{$\mathcal{S}$\xspace}
\newcommand{\SH}		{$\mathcal{SH}$\xspace}
\newcommand{\SI}		{$\mathcal{SI}$\xspace}
\newcommand{\SHI}		{$\mathcal{SHI}$\xspace}
\newcommand{\SHIQ}		{$\mathcal{SHIQ}$\xspace}
\newcommand{\SHIN}		{$\mathcal{SHIN}$\xspace}
\newcommand{\SHIO}		{$\mathcal{SHIO}$\xspace}
\newcommand{\SHOQ}		{$\mathcal{SHOQ}$\xspace}
\newcommand{\SHOIN}		{$\mathcal{SHOIN}$\xspace}
\newcommand{\SHOIQ}		{$\mathcal{SHOIQ}$\xspace}
\newcommand{\SROIQ}		{$\mathcal{SROIQ}$\xspace}

\newcommand{\Ref}			{\mathtt{Ref}}
\newcommand{\Irr}			{\mathtt{Irr}}
\newcommand{\Sym}			{\mathtt{Sym}}
\newcommand{\Tra}			{\mathtt{Tra}}
\newcommand{\Dis}			{\mathtt{Dis}}
\newcommand{\BBCLearn}		{BBCL\xspace}
\newcommand{\BBCLearnS}		{BBCL2\xspace}
\newcommand{\dualBBCLearn}	{dual-BBCL\xspace}
\newcommand{\mdepth}		{\textsf{mdepth}}
\newcommand{\length}		{\textsf{length}}
\renewcommand{\sharp}		{\#}

\newcommand{\HRule}{\rule{\linewidth}{0.6mm}}
%\newcommand{\myend}			{\mbox{}\hfill\mbox{{\scriptsize$\!\blacksquare$}}}
\newcommand{\myend}			{\mbox{}\hfill\mbox{{\footnotesize$\!\square$}}}
%\newcommand{\myend}			{\mbox{}\hfill\mbox{{\footnotesize$\!\Box$}}}
\renewcommand{\qedsymbol}	{\myend}
%\newenvironment{sketch}{\noindent{\em Proof sketch.}}{\myend\smallskip}

\newcommand{\semiItem}		{\mbox{- }}
\newcommand{\semiBullet}[1] {\mbox{$\bullet$ {\bf {#1}}}}
\newcommand{\tuple}[1]		{\left\langle#1\right\rangle\!}
\newcommand{\ramka}[1]		{\fbox{\parbox{\textwidth}{#1}}}
\newcommand{\mand}			{\sqcap}
\newcommand{\mor}			{\sqcup}
\newcommand{\V}				{\forall}
\newcommand{\E}				{\exists}
\newcommand{\Range}			{\mathit{range}}

\newcommand{\PTIME}			{{\sc PTime}\xspace}
\newcommand{\PSPACE}		{{\sc PSpace}\xspace}
\newcommand{\NP}			{{\sc NP}\xspace}
\newcommand{\EXPTIME}		{{\sc ExpTime}\xspace}
\newcommand{\NEXPTIME}		{{\sc NExpTime}\xspace}
\newcommand{\NdEXPTIME}		{{\sc N2ExpTime}\xspace}
\newcommand{\NtEXPTIME}		{{\sc N3ExpTime}\xspace}

%------------------------------------------------------------
%Các ký hiệu cho ví dụ (khái niệm + vai trò)
\newcommand{\Human}			{Human}
\newcommand{\Female}		{Female}
\newcommand{\Male}			{Male}
\newcommand{\Rich}			{Rich}
\newcommand{\Parent}		{Parent}
\newcommand{\Mother}		{Mother}
\newcommand{\Father}		{Father}
\newcommand{\Husband}		{Husband}
\newcommand{\Niece}			{Niece}
\newcommand{\Nephew}		{Nephew}
\newcommand{\BirthYear}		{BirthYear}
\newcommand{\NickName}		{NickName}

\newcommand{\hasChild}		{hasChild}
\newcommand{\hasParent}		{hasParent}
\newcommand{\hasSibling}	{hasSibling}
\newcommand{\marriedTo}		{marriedTo}
\newcommand{\hasDescendant}	{hasDescendant}
\newcommand{\hasAscendant}	{hasAscendant}

\newcommand{\Publication}	{\mathit{Pub}}
\newcommand{\Pub}			{\mathit{P}}
\newcommand{\Book}			{\mathit{Book}}
\newcommand{\Article}		{\mathit{Article}}
\newcommand{\Kind}			{\mathit{Kind}}
\newcommand{\Awarded}		{\mathit{Awarded}}
\newcommand{\PubName}		{\mathit{Title}}
\newcommand{\PubYear}		{\mathit{Year}}
\newcommand{\Cites}			{\mathit{cites}}
\newcommand{\Citedby}		{\mathit{cited\!\_\!by}}
\newcommand{\UsefulPub}		{\mathit{UsefulPub}}
\newcommand{\GoodPub}		{\mathit{GoodPub}}
\newcommand{\ExcellentPub}	{\mathit{ExcellentPub}}
\newcommand{\RecentPub}		{\mathit{RecentPub}}
\newcommand{\CitingP}		{\mathit{CitingP}}
\newcommand{\textItL}		{\textrm{``Introduction to Logic''}}
\newcommand{\textTEoL}		{\textrm{``The Essence of Logic''}}
\newcommand{\textB}			{\textrm{``book''}}
\newcommand{\textA}			{\textrm{``article''}}
\newcommand{\textC}			{\textrm{``conf.~paper''}}

%-------------------------------------------------------------
%Các ký hiệu cho ví dụ (cá thể)
\newcommand{\iLAN}			{\mathsf{LAN}}
\newcommand{\iHUNG}			{\mathsf{HUNG}}
\newcommand{\iHAI}			{\mathsf{HAI}}

\newcommand{\iALICE}		{\mathsf{ALICE}}
\newcommand{\iBOB}			{\mathsf{BOB}}
\newcommand{\iCLAUDIA}		{\mathsf{CLAUDIA}}
\newcommand{\iCALVIN}		{\mathsf{CALVIN}}

\newcommand{\iANH}			{\mathsf{ANH}}
\newcommand{\iPHAP}			{\mathsf{PHAP}}
\newcommand{\iMY}			{\mathsf{MY}}
\newcommand{\iNGA}			{\mathsf{NGA}}
\newcommand{\iTRUNGQUOC}	{\mathsf{TRUNGQUOC}}

%----------------------------------------------------------
%Định dạng chương, định lý, mệnh đề, ...
\newtheorem{Theorem}{Định lý}[chapter]
\newtheorem{Proposition}{Mệnh đề}[chapter]
\newtheorem{Lemma}{Bổ đề}[chapter]
\newtheorem{Corollary}{Hệ quả}[chapter]
\newtheorem{Remark}{Ghi chú}[chapter]

\theoremstyle{definition}
\newtheorem{Definition}{Định nghĩa}[chapter]
\newtheorem{Example}{Ví dụ}[chapter]

%Định nghĩa các màu dùng cho các tiêu đề CHƯƠNG, MỤC
\definecolor{colorChapter}{rgb}{0.00, 0.00, 0.00}
\definecolor{colorSection}{rgb}{0.00, 0.00, 0.00}
%\definecolor{colorChapter}{rgb}{0.81, 0.09, 0.13}
%\definecolor{colorSection}{rgb}{0.16, 0.32, 0.95}
%-----------------------------------------------------------------
%Thiết lập các font cho tiêu đề CHƯƠNG, MỤC, TIỂU MỤC
\newcommand{\chapterFont}{\normalfont\fontsize{16}{15}\bfseries\color{colorChapter}}
\newcommand{\sectionFont}{\normalfont\fontsize{13}{15}\bfseries\color{colorSection}}
\newcommand{\subsectionFont}{\normalfont\fontsize{12}{15}\bfseries\color{black}}
\newcommand{\subsubsectionFont}{\normalfont\fontsize{12}{15}\bfseries\color{black}}
%-----------------------------------------------------------------
%Thiết lập thông số các tiêu đề của CHƯƠNG, MỤC
\setcounter{secnumdepth}{3}
\titleformat{\chapter}[display]{\chapterFont}{\chaptertitlename\ \thechapter.}{10pt}{\centering} % display -> block/hange
\titlespacing{\chapter}{0pt}{0pt}{25pt} % left, before, after

\titleformat{\section}{\sectionFont}{\thesection.}{10pt}{}
\titlespacing{\section}{0pt}{6pt}{2pt}

\titleformat{\subsection}{\subsectionFont}{\thesubsection.}{10pt}{}
\titlespacing{\subsection}{0pt}{5pt}{2pt}

\titleformat{\subsubsection}{\subsubsectionFont}{\thesubsubsection.}{10pt}{}
\titlespacing{\subsubsection}{0pt}{5pt}{2pt}

%-----------------------------------------------------------------
%\usepackage{draftwatermark}
%\SetWatermarkText{version seminar}
%\SetWatermarkLightness{0.9}
%\SetWatermarkScale{0.65}
%-----------------------------------------------------------------
%Thiết lập thông số mục lục cho CHƯƠNG
\newlength\myLenChap
\renewcommand\cftchappresnum{\chaptername~}
\renewcommand\cftchapaftersnum{.}
\settowidth\myLenChap{\cftchappresnum\cftchapaftersnum}
\addtolength\cftchapnumwidth{\myLenChap}
%-----------------------------------------------------------------
%Thiết lập thông số mục lục cho MỤC
\newlength\myLenSec
\renewcommand\cftsecaftersnum{.}
\settowidth\myLenSec{\cftsecpresnum\!}
\addtolength\cftsecnumwidth{\myLenSec}
%-----------------------------------------------------------------
%Thiết lập thông số mục lục cho TIỂU MỤC
\newlength\myLenSubsec
\renewcommand\cftsubsecaftersnum{.}
\settowidth\myLenSubsec{\cftsubsecpresnum\!\!}
\addtolength\cftsubsecnumwidth{\myLenSubsec}
%-----------------------------------------------------------------
%Thiết lập thông số mục lục cho HÌNH ẢNH
\setlength{\cftfigindent}{0pt}
\newlength{\myLenFig}
\renewcommand{\cftfigpresnum}{\figurename\;}
\renewcommand{\cftfigaftersnum}{.}
\settowidth{\myLenFig}{\cftfigpresnum\!\!\!\!\!}
\addtolength{\cftfignumwidth}{\myLenFig}
%-----------------------------------------------------------------
%Thiết lập thông số mục lục cho BẢNG BIỂU
\setlength{\cfttabindent}{0pt}
\newlength{\myLenTab}
\renewcommand{\cfttabpresnum}{\tablename\;}
\renewcommand{\cfttabaftersnum}{.}
\settowidth{\myLenTab}{\cfttabpresnum\!\!}
\addtolength{\cfttabnumwidth}{\myLenTab}

\makeatletter
\def\cleardoublepage
{
	\clearpage\if@twoside \ifodd\c@page\else
	\hbox{}
	\vspace*{\fill}
%	\begin{center}
%		{\em This page intentionally left blank}
%	\end{center}
	\vspace{\fill}
	\begin{flushleft}
		\noindent\rule{12cm}{1pt}\\[-0.25cm]
		 \thepage\;- \textsc{đề tài nghiên cứu khoa học cấp cơ sở đại học huế}
	\end{flushleft}
	\thispagestyle{empty}
	\newpage
	\if@twocolumn\hbox{}\newpage\fi\fi\fi
}