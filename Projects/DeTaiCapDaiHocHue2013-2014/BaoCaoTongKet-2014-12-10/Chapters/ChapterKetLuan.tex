\chapter*{KẾT LUẬN}
\label{ChapterKetLuan}
\addcontentsline{toc}{chapter}{Kết luận}
\thispagestyle{fancy}
%----------------------------------------------------------------------------------------
\section*{Kết luận}
Kể từ khi logic mô tả được xem là nền tảng của ngôn ngữ OWL (một ngôn ngữ được sử dụng để mô hình hóa các hệ thống ngữ nghĩa và ontology theo khuyến nghị của W3C), logic mô tả đã được nhiều nhà khoa học quan tâm nghiên cứu. Đối với các hệ thống ngữ nghĩa, việc xây dựng những định nghĩa cho các khái niệm phù hợp để đặc tả hệ thống là một vấn đề được đặt ra một cách tự nhiên. Học khái niệm trong logic mô tả là một trong những giải pháp để tìm kiếm và xây dựng định nghĩa cho các khái niệm. Với mục đích đó, đề tài nghiên cứu bài toán học khái niệm cho cơ sở tri thức trong logic mô tả với ngữ cảnh~(2). Kết quả nghiên cứu của đề tài được tóm tắt như sau:
\begin{enumerate}
	\item Xây dựng ngôn ngữ logic mô tả $\mLSP$ dựa trên ngôn ngữ \ALCreg với tập các đặc trưng mở rộng gồm $\mI$, $\mO$, $\mN$, $\mQ$, $\mF$, $\mU$, $\Self$. Ngoài ra ngôn ngữ được xây dựng còn cho phép sử dụng các thuộc tính (bao gồm thuộc tính rời rạc và liên tục) như là các phần tử cơ bản của ngôn ngữ nhằm mô tả các hệ thống thông tin phù hợp với thực tế hơn.
	
	\item Xây dựng mô phỏng hai chiều trên lớp các logic mở rộng đang nghiên cứu. Các định lý, bổ đề, hệ quả liên quan đến mô phỏng hai chiều và tính bất biến đối với mô phỏng hai chiều cũng được phát triển và chứng minh trên lớp các logic mở rộng này.
	
	\item Dựa vào mô phỏng hai chiều, xây dựng thuật toán phân hoạch miền của mô hình của cơ sở tri thức và xây dựng thuật toán \BBCLearnS để học khái niệm trong logic mô tả cho cơ sở tri thức với ngữ cảnh~(2).
\end{enumerate}

%----------------------------------------------------------------------------------------
\section*{Những vấn đề cần tiếp tục nghiên cứu}

\begin{enumerate}
	\item Xây dựng các chiến lược học khác nhau thông qua các độ đo trong việc quyết định khối nào nên phân hoạch trước. So sánh các chiến lược học với nhau.
	
	\item Xây dựng các module học khái niệm trong logic mô tả với các ngữ cảnh khác 	nhau như là một API cho phép tích hợp vào các hệ thống khác.
	
	\item Nghiên cứu khả năng học chính xác khái niệm cho các logic mô tả khác nhau.
\end{enumerate}
%----------------------------------------------------------------------------------------
\cleardoublepage