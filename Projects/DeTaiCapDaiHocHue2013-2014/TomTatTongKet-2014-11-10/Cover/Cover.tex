
\begin{titlepage}
	\begin{tikzpicture}[remember picture, overlay]
	\draw[line width = 4pt] ($(current page.north west) + (2.5cm,-2.3cm)$) rectangle ($(current page.south east) + (-2.0cm, 2.4cm)$);
	\end{tikzpicture}
	
	\begin{adjustwidth}{-10pt}{-10pt}
		\begin{center}
		\vspace{-0.3cm}
			\textsc{\textbf{BỘ GIÁO DỤC VÀ ĐÀO TẠO}}\\[0.0cm]
			\textsc{\textbf{ĐẠI HỌC HUẾ}}\\[0.0cm]
			{\textbf{TRƯỜNG ĐẠI HỌC KHOA HỌC}}\\[5.0cm]
			
					
			\textsc{\Large \textbf{TÓM TẮT BÁO CÁO TỔNG KẾT}}\\[0.3cm]
			\textsc{\large \textbf{ĐỀ TÀI NGHIÊN CỨU KHOA HỌC CẤP CƠ SỞ}}\\[1.5cm]
			% Title
%			\HRule \\[0.1cm]
			{\fontsize{14.5}{15} \bfseries HỌC KHÁI NIỆM ĐỐI VỚI CÁC CƠ SỞ TRI THỨC\\
			TRONG LOGIC MÔ TẢ DỰA VÀO MÔ PHỎNG HAI CHIỀU}\\[0.4cm]
%			\HRule \\[0.3cm]
			\textbf{Mã số: DHH2013-01-41}
			\\[1.5cm]
	
			\textbf{Chủ nhiệm đề tài: ThS. TRẦN THANH LƯƠNG}\\[3.2cm]
			
			\vfill
			% Bottom of the page
			Thừa Thiên Huế, 11/2014
			\vspace{0.1cm}
		\end{center}
	\end{adjustwidth}
\end{titlepage}
\cleardoublepage

%-----------------------------------------------------------------------------
%\newpage
%\vspace*{\fill}
%\begin{center}
%	\textcolor{white}{.}
%\end{center}
%\vspace{\fill}
%\thispagestyle{empty}
%\clearpage

%-----------------------------------------------------------------------------
\begin{titlepage}
	\begin{tikzpicture}[remember picture, overlay]
	\draw[line width = 4pt] ($(current page.north west) + (2.5cm,-2.3cm)$) rectangle ($(current page.south east) + (-2.0cm, 2.4cm)$);
	\end{tikzpicture}

	\begin{adjustwidth}{-10pt}{-10pt}
	\begin{center}
		\vspace{-0.3cm}		
		\textsc{\textbf{BỘ GIÁO DỤC VÀ ĐÀO TẠO}}\\[0.0cm]
		\textsc{\textbf{ĐẠI HỌC HUẾ}}\\[0.0cm]
		{\textbf{TRƯỜNG ĐẠI HỌC KHOA HỌC}}\\[5.0cm]
				
		\textsc{\Large \textbf{TÓM TẮT BÁO CÁO TỔNG KẾT}}\\[0.3cm]
		\textsc{\large \textbf{ĐỀ TÀI NGHIÊN CỨU KHOA HỌC CẤP CƠ SỞ}}\\[1.5cm]
		% Title
%		\HRule \\[0.1cm]
		{\fontsize{14.5}{15} {\bf HỌC KHÁI NIỆM ĐỐI VỚI CÁC CƠ SỞ TRI THỨC\\
		TRONG LOGIC MÔ TẢ DỰA VÀO MÔ PHỎNG HAI CHIỀU}}\\[0.5cm]
%		\HRule \\[0.3cm]
		\textbf{Mã số: DHH2013-01-41}
		\\[1.5cm]

		\begin{minipage}{0.45\textwidth}
		\begin{flushleft}
		{\bf Xác nhận của cơ quan chủ trì đề tài}\\[2.5cm]
		{\ }
		\end{flushleft}
		\end{minipage}
		\begin{minipage}{0.45\textwidth}
		\begin{center}
		{\bf Chủ nhiệm đề tài} \\[2.5cm]
		{ThS. TRẦN THANH LƯƠNG}\\
		\end{center}
		\end{minipage}		
		\vfill
		% Bottom of the page
		Thừa Thiên Huế, 11/2014
		\vspace{0.1cm}
	\end{center}
	\end{adjustwidth}
\end{titlepage}
%\cleardoublepage

%-----------------------------------------------------------------------------
\newpage
\thispagestyle{empty}
~

\fontsize{13.5}{17}\selectfont
\textbf{DANH SÁCH THÀNH VIÊN THAM GIA NGHIÊN CỨU:}

\textbf{$\bullet$ TS. Hoàng Thị Lan Giao}, 
\vspace{-1.0ex}

Khoa Công nghệ Thông tin, Trường Đại học Khoa học, Đại học Huế
\vspace{-1.0ex}

77 Nguyễn Huệ, Thành phố Huế, Tỉnh Thừa Thiên Huế, Việt Nam

\vspace{3.0ex}
\textbf{ĐƠN VỊ PHỐI HỢP NGHIÊN CỨU:}

\textbf{$\bullet$ Viện Tin học},
\vspace{-1.0ex}

Khoa Toán - Tin học và  Cơ học, Trường Đại học Tổng hợp Vác-xa-va, Ba Lan

\vspace{-1.0ex}
Banacha 2, 02-097 Warsaw, Poland
%-----------------------------------------------------------------------------
\cleardoublepage

%-----------------------------------------------------------------------------
%\newpage
%\chapter*{KÝ HIỆU VIẾT TẮT}
%\addcontentsline{toc}{chapter}{Ký hiệu viết tắt}
%
%\begin{tabular}{| c | p{2.3cm} | p{4.5cm} | p{4.5cm}|}
%	\hline
%	{\bf STT} & \multicolumn{1}{c|}{\bf Từ viết tắt} & \multicolumn{1}{c|}{\bf Từ Tiếng Việt đầy đủ} & \multicolumn{1}{c|}{\bf Từ Tiếng Anh đầy đủ}\\
%	\hline
%	& NFF & Dạng chuẩn phủ định & Negation Normal Form\\
%	& CNF & Dạng chuẩn nghịch đảo & Converse Normal Form\\
%	& RBox & Bộ tiên đề vai trò & Role Box\\
%	& TBox & Bộ tiên đề thuật ngữ & Terminological Box\\
%	& ABox & Bộ khẳng định cá thể & Assertion Box\\
%	& CWA & Giả thiết thế giới đóng & Close World Assumption\\
%	& OWA & Giả thiết thế giới mở & Open World Assumption\\
%	\hline
%\end{tabular}
%\clearpage

%-----------------------------------------------------------------------------
\newpage
\fontsize{13.5}{15}\selectfont
\pagenumbering{roman}
\setcounter{tocdepth}{2}
\addcontentsline{toc}{chapter}{Mục lục}
\renewcommand\contentsname{\!\!\!\!\!\!\!\!\chapterFont\hfill MỤC LỤC\hfill}
\setlength\cftparskip{-1pt}
\setlength\cftbeforechapskip{0pt}
\tableofcontents{\thispagestyle{plain}}
\cleardoublepage
%-----------------------------------------------------------------------------
%\newpage
%\renewcommand\listtablename{\!\!\!\!\!\!\!\!\!\chapterFont\hfill DANH MỤC BẢNG, BIỂU\hfill}
%\addcontentsline{toc}{chapter}{Danh mục bảng, biểu}
%\listoftables
%\clearpage

%-----------------------------------------------------------------------------
%\newpage
%\renewcommand\listfigurename{\!\!\!\!\!\!\!\!\!\chapterFont\hfill DANH MỤC HÌNH VẼ\hfill}
%\addcontentsline{toc}{chapter}{Danh mục hình vẽ}
%\listoffigures{\thispagestyle{fancy}}
%-----------------------------------------------------------------------------
%\cleardoublepage

%-----------------------------------------------------------------------------
\newpage
\chapter*{THÔNG TIN KẾT QUẢ NGHIÊN CỨU}
\addcontentsline{toc}{chapter}{Thông tin kết quả nghiên cứu}
\thispagestyle{fancy}

\section*{1. Thông tin chung:}
\begin{itemize}
	\item Tên đề tài:~~\textbf{HỌC KHÁI NIỆM ĐỐI VỚI CÁC CƠ SỞ TRI THỨC\\
	\mbox{~}\hspace{7ex} TRONG LOGIC MÔ TẢ DỰA VÀO MÔ PHỎNG HAI CHIỀU}\\[-0.8cm]
	\item Mã số: {\bf DHH-2013-01-41}\\[-0.8cm]
	\item Chủ nhiệm đề tài: {\bf ThS. Trần Thanh Lương}
	
	\vspace{-0.05cm}
	Điện thoại: \texttt{091 4145414} \qquad\qquad\qquad\qquad E-mail: \texttt{ttluong@hueuni.edu.vn}\\[-0.8cm]
	\item Cơ quan chủ trì đề tài: {\bf Trường Đại học Khoa học, Đại học Huế}\\[-0.8cm]
	\item Cơ quan và cá nhân phối hợp thực hiện:
	
	{\bf - TS. Hoàng Thị Lan Giao,}
	
%	\vspace{-0.05cm}	
	Khoa Công nghệ Thông tin Trường Đại học Khoa học, Đại học Huế
	
%	\vspace{-0.05cm}	
	77 Nguyễn Huệ, Thành phố Huế, Tỉnh Thừa Thiên Huế, Việt Nam
	
	{\bf - Viện Tin học,}
	
%	\vspace{-0.05cm}	
	Khoa Toán, Tin học và Cơ học, Trường Đại học Tổng hợp Vác-xa-va, Ba Lan
	
%	\vspace{-0.05cm}	
	Banacha 2, 02-097 Warsaw, Poland\\[-0.8cm]
	
	\item Thời gian thực hiện: Từ {\bf tháng 01 năm 2013} đến {\bf tháng 12 năm 2014}
\end{itemize}
\section*{2. Mục tiêu}
Mở rộng lý thuyết mô phỏng hai chiều và phương pháp học khái niệm cho cơ sở tri thức trong logic mô tả với một số điều kiện cho trước.
Đề xuất thuật toán học khái niệm dựa trên mô phỏng hai chiều để giải quyết bài toán phân lớp các đối tượng trong logic mô tả.

\section*{3. Tính mới và sáng tạo}
Xây dựng mô phỏng hai chiều để mô hình hóa tính không phân biệt của các đối tượng được trên một lớp lớn các logic mô tả. Từ đó đề xuất thuật toán phân hoạch miền và học khái niệm cho cơ sở tri thức trong logic mô tả sử dụng mô phỏng hai~chiều.

\section*{4. Kết quả nghiên cứu}
%Báo cáo đề tài với những kết quả cụ thể sau:
\begin{itemize}
	\item Xây dựng ngôn ngữ $\mLSP$ dựa trên logic mô tả \ALCreg với tập các đặc trưng logic mô tả mở rộng gồm $\mI$, $\mO$, $\mN$, $\mQ$, $\mF$, $\mU$, $\Self$. Ngoài ra ngôn ngữ được xây dựng còn cho phép sử dụng các thuộc tính (mỗi thuộc tính có thể là rời rạc hoặc số) như là các phần tử cơ bản của ngôn ngữ. Cách tiếp cận này rất phù hợp đối với các hệ thống thông tin trong thực~tế.
	
	\item Xây dựng mô phỏng hai chiều trên lớp các logic mở rộng đang nghiên cứu. Các định lý, bổ đề, hệ quả, mệnh đề liên quan đến mô phỏng hai chiều và tính bất biến đối với mô phỏng hai chiều cũng được phát triển và chứng minh trên lớp các logic mở rộng này.
	
	\item Dựa vào mô phỏng hai chiều, xây dựng thuật toán để phân hoạch miền của mô hình của cơ sở tri thức và thuật toán \BBCLearnS để học khái niệm cho cơ sở tri thức trong logic mô tả.
\end{itemize}
\section*{5. Sản phẩm}
\begin{itemize}
	\item Hướng dẫn 01 luận văn Thạc sĩ Khoa học chuyên ngành Khoa học Máy tính\\
	(Người hướng dẫn: TS. Hoàng Thị Lan Giao, thành viên đề tài).
	\item Hướng dẫn 02 khóa luận Tốt nghiệp Đại học ngành Tin học\\
	(Người hướng dẫn: ThS. Trần Thanh Lương, chủ trì đề tài).
	\item Công bố 04 bài báo trên các tạp chí/hội thảo khoa học trong nước và quốc tế.
	\item Báo cáo đề tài.
\end{itemize}
\section*{6. Hiệu quả, phương thức chuyển giao kết quả nghiên cứu và khả năng áp~dụng}
\begin{itemize}
	\item Phương pháp học khái niệm cho cơ sở tri thức trong logic mô tả sử dụng mô phỏng hai chiều có thể áp dụng để tìm kiếm, xây dựng các định nghĩa khái của các niệm phù hợp cho hệ thống ngữ nghĩa nói chung và Web ngữ~nghĩa nói riêng.

	\item Báo cáo làm tài liệu tham khảo cho sinh viên đại học, học viên cao học và những người nghiên cứu trong chuyên ngành Khoa học Máy tính nói chung cũng như logic mô tả và học máy nói riêng.

	\item Địa chỉ ứng dụng: Khoa Công nghệ Thông tin, Khoa Tin học của các trường Đại học trong cả~nước.
\end{itemize}
%---------------------------------------------------------------------
\begin{minipage}{0.50\textwidth}
	\begin{flushleft}
		{~\;}\\[0.3cm]
		{\bf~~~~~~~~~~Cơ quan chủ trì} \\[2.4cm]
		{~\;}
	\end{flushleft}
\end{minipage}
\begin{minipage}{0.50\textwidth}
	\begin{center}
		\textit{Ngày 25 tháng 11 năm 2014}\\
		{\bf Chủ nhiệm đề tài} \\[2.4cm]
		{\bf ThS. TRẦN THANH LƯƠNG}
	\end{center}
\end{minipage}		

%\cleardoublepage
%---------------------------------------------------------------------
\newpage
\chapter*{INFORMATION ON RESEARCH RESULTS}
\addcontentsline{toc}{chapter}{Information on research results}
\thispagestyle{fancy}

\section*{1. General information}
\begin{itemize}
	\item Project title:~~\textbf{CONCEPT LEARNING FOR KNOWLEDGE BASES\\ \mbox{~}\hspace{12.5ex} IN DESCRIPTION LOGIC USING BISIMULATION}\\[-0.8cm]
	\item Code number: {\bf DHH-2013-01-41}\\[-0.8cm]
	\item Coordinator: {\bf MSc Tran Thanh Luong}\\[-0.8cm]
	\item Implementing institution: {\bf College of Sciences, Hue University}\\[-0.8cm]
	\item Cooperating institution(s):
	
	{\bf - Dr. Hoang Thi Lan Giao,}\\
	Department of Information Technology, College of Sciences, Hue University\\
	77 Nguyen Hue, Hue City, Thua Thien Hue Province, Vietnam
	
	{\bf - Institution of Informatics,}\\
	Faculty of Mathematics, Informatics and Mechanics, Warsaw University, Poland
	Banacha 2, 02-097 Warsaw, Poland\\[-0.8cm]
	\item Duration: from {\bf January 2013} to {\bf December 2014}
\end{itemize}
\section*{2. Objective(s)}	
\begin{itemize}
	\item Extend the theory of bisimulation and method of concept learning for knowledge bases in description logics (DLs) using given conditions. 

	\item Propose a bisimulation-based concept learning algorithm for classifying objects in DLs.
\end{itemize}
\section*{3. Creativeness and innovativeness}
We built bisimulation to model indiscernibility of objects in a large class of DLs. We also proposed algorithms for partitioning and leaning concepts for knowledge bases in DLs using bisimulation.
\section*{4. Research results}
A report consists of information about:
\begin{itemize}
	\item Consider the language $\mLSP$, where $\mL$ stands for \ALCreg, with the set of DL-features, including $\mI$, $\mO$, $\mN$, $\mQ$, $\mF$, $\mU$, $\Self$. In addition, this language allows to use attributes as basic elements (each attribute may be discrete or numeric). This approach is suitable for practical information systems
	based on DLs.
	
	\item Study bisimulation for the considered class of DLs. Theorems, lemmas, corollaries, propositions related to bisimulations and invariant results for bisimulation are developed and proved for the considered class of DLs.
	
	\item Develop bisimulation-based algorithms to partition the domain of model of knowledge bases and learn concepts for knowledge bases in DLs (Algorithm \BBCLearnS).
\end{itemize}
\section*{5. Products}
\begin{itemize}
	\item 01 Master thesis, major: Computer Science\\
	(Supervisor: Dr. Hoang Thi Lan Giao, a co-applicant)
	\item 02 Bachelor theses, major: Informatics\\
	(Supervisor: Msc. Tran Thanh Luong, the coordinator)
	\item 04 papers published national/international journals/conferences
	\item A report of project.
\end{itemize}

\section*{6. Effects, transfer alternatives of research results and applicability}
Bisimulation-based concept learning method for knowledge bases in DLs can be applied to find and build definitions of suitable concepts for ontologies and Semantic Webs.

This project is a material for students and researchers in the computer science major in general as well as DLs and machine learning in particular.

Application address: Department of Information Technology, Department of Informatics and Department of Computer Science in Universities of Vietnam.

\cleardoublepage