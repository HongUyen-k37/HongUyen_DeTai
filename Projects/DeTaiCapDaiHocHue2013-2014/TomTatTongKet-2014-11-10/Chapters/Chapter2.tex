\chapter[Mô phỏng hai chiều trong logic mô tả và tính bất biến]{MÔ PHỎNG HAI CHIỀU TRONG LOGIC MÔ TẢ\\ VÀ TÍNH BẤT BIẾN}
\label{Chapter2}
\thispagestyle{fancy}

%-------------------------------------------------------------------
\section{Giới thiệu}
\label{sec:Chap2.Introduction}
Mô phỏng hai chiều được J. van Benthem giới thiệu lần đầu dưới tên gọi {\em p-quan hệ} ({\em p-relation}) và {\em quan hệ zig-zag} ({\em zig-zag relation}). Nó được phát triển trong logic hình thái ({\em modal logic}) và trong các hệ thống chuyển trạng thái ({\em state transition systems}). 
Mô phỏng hai chiều là một quan hệ nhị phân cho phép đặc tả tính tương tự giữa hai trạng thái cũng như tính tương tự giữa các mô hình Kripke~\cite{Benthem1984,Benthem2010,Benthem2001,Blackburn2001}. Divroodi và Nguyễn đã phát triển logic mô tả hai chiều trong logic mô tả \ALCreg với tập các đặc trưng là $\mI, \mO, \mQ, \mU, \Self$~\cite{Divroodi2011B}. Chúng tôi mở rộng mô phỏng hai chiều cho một lớp lớn hơn các logic mô tả với các đặc trưng $\mF, \mN$. Bên cạnh đó, chúng tôi cũng đề cập đến các thuộc tính như là các phần tử cơ bản của ngôn ngữ cần xem xét.
%-------------------------------------------------------------------
\section{Mô phỏng hai chiều trong logic mô tả}
\label{sec:Chap2.BisimulationInDL}
\subsection{Mô phỏng hai chiều}
\label{sec:Chap2.Bisimulation}

\begin{Definition}[Mô phỏng hai chiều]
	\label{def:Bisimulation}
	Cho $\Sigma$ và $\SigmaDag$ là các bộ ký tự logic mô tả sao cho $\SigmaDag \subseteq \Sigma$, $\Phi$ và $\PhiDag$ là tập các đặc trưng của logic mô tả sao cho $\PhiDag \subseteq \Phi$, $\mI$ và $\mI'$ là các diễn dịch trong $\mLSP$.
	%
	Một {\em $\mLSPD$-mô phỏng hai chiều} giữa $\mI$ và $\mI'$ là một quan hệ nhị phân $Z \subseteq \Delta^\mI \times \Delta^{\mI'}$ thỏa các điều kiện sau với mọi $a \in \SigmaDagI$, $A \in \SigmaDagC$, $B \in \SigmaDagA\setminus\SigmaDagC$, $r \in \SigmaDagOR$, $\sigma \in \SigmaDagDR$, $d \in \Range(\sigma)$, $x,y \in \Delta^\mI$, $x',y' \in \Delta^{\mI'}$:	
	\begin{eqnarray}
		&&\!\!\!\!\!\!\!\!\!\!\!\!\!\!\!\!\!\!\!\!\!\!\!\!
		Z(a^\mI,a^{\mI'}) \label{bs:eqA} \\
		&&\!\!\!\!\!\!\!\!\!\!\!\!\!\!\!\!\!\!\!\!\!\!\!\!
		Z(x,x') \Rightarrow [A^\mI(x) \Leftrightarrow A^{\mI'}(x')] \label{bs:eqB1} \\
		&&\!\!\!\!\!\!\!\!\!\!\!\!\!\!\!\!\!\!\!\!\!\!\!\!
		Z(x,x') \Rightarrow [B^\mI(x) = B^{\mI'}(x') \textrm{ hoặc đều không xác định}] \label{bs:eqB2} \\
		&&\!\!\!\!\!\!\!\!\!\!\!\!\!\!\!\!\!\!\!\!\!\!\!\!
		[Z(x,x') \wedge r^\mI(x,y)] \Rightarrow \E y' \in \Delta^{\mI'} \mid [Z(y,y') \wedge r^{\mI'}(x',y')] \label{bs:eqC1}\\
		&&\!\!\!\!\!\!\!\!\!\!\!\!\!\!\!\!\!\!\!\!\!\!\!\!
		[Z(x,x') \wedge r^{\mI'}(x',y')] \Rightarrow \E y \in \Delta^\mI \mid [Z(y,y') \wedge r^\mI(x,y)] \label{bs:eqC2} \\
		&&\!\!\!\!\!\!\!\!\!\!\!\!\!\!\!\!\!\!\!\!\!\!\!
		Z(x,x') \Rightarrow [\sigma^\mI(x,d) \Leftrightarrow \sigma^{\mI'}(x',d)], \label{bs:eqD}
	\end{eqnarray}
	%
	nếu $\mI \in \Phi^\dag$ thì
	\begin{eqnarray}
		&&\!\!\!\!\!\!\!\!\!\!\!\!\!\!\!\!\!\!\!\!\!\!\!\!
		[Z(x,x') \wedge r^\mI(y,x)] \Rightarrow \E y' \in \Delta^{\mI'} \mid [Z(y,y') \wedge r^{\mI'}(y',x')] \label{bs:eqI1} \\
		&&\!\!\!\!\!\!\!\!\!\!\!\!\!\!\!\!\!\!\!\!\!\!\!\!
		[Z(x,x') \wedge r^{\mI'}(y',x')] \Rightarrow \E y \in \Delta^\mI \mid [Z(y,y') \wedge r^\mI(y,x)], \label{bs:eqI2}
	\end{eqnarray}
	%
	nếu $\mO \in \Phi^\dag$ thì
	\begin{eqnarray}
		&&\!\!\!\!\!\!\!\!\!\!\!\!\!\!\!\!\!\!\!\!\!\!\!\!
		Z(x,x') \Rightarrow [x = a^\mI \Leftrightarrow x' = a^{\mI'}], \label{bs:eqO0}
	\end{eqnarray}
	%
	nếu $\mN \in \Phi^\dag$ thì
	\begin{eqnarray}
		&&\!\!\!\!\!\!\!\!\!\!\!\!\!\!\!\!\!\!\!\!\!\!\!\!
		Z(x,x') \Rightarrow \#\{y \in \Delta^\mI \mid r^\mI(x,y)\} = \#\{y' \in \Delta^{\mI'} \mid r^{\mI'}(x',y')\}, \label{bs:eqN}
	\end{eqnarray}
	%
	nếu $\{\mN,\mI\} \subseteq \Phi^\dag$ thì
	\begin{eqnarray}
		&&\!\!\!\!\!\!\!\!\!\!\!\!\!\!\!\!\!\!\!\!\!\!\!\!
		Z(x,x') \Rightarrow \#\{y \in \Delta^\mI \mid r^\mI(y,x)\} = \#\{y' \in \Delta^{\mI'} \mid r^{\mI'}(y',x')\}, \label{bs:eqNI}
	\end{eqnarray}
	%
	nếu $\mF \in \Phi^\dag$ thì
	\begin{eqnarray}
		&&\!\!\!\!\!\!\!\!\!\!\!\!\!\!\!\!\!\!\!\!\!\!\!\!
		\begin{array}{c}
			Z(x,x') \Rightarrow [\#\{y \in \Delta^\mI \mid r^\mI(x,y)\} \leq 1 \Leftrightarrow \#\{y' \in \Delta^{\mI'} \mid r^{\mI'}(x',y')\} \leq 1],
		\end{array}\label{bs:eqF}
	\end{eqnarray}
	%
	nếu $\{\mF,\mI\} \subseteq \Phi^\dag$ thì
	\begin{eqnarray}
	&&\!\!\!\!\!\!\!\!\!\!\!\!\!\!\!\!\!\!\!\!\!\!\!\!
		\begin{array}{c}
			Z(x,x') \Rightarrow [\#\{y \in \Delta^\mI \mid r^\mI(y,x)\} \leq 1 \Leftrightarrow \#\{y' \in \Delta^{\mI'} \mid r^{\mI'}(y',x')\} \leq 1],
		\end{array}	\label{bs:eqFI}
	\end{eqnarray}
	%
	nếu $\mQ \in \Phi^\dag$ thì
	\begin{eqnarray}
	&&\!\!\!\!\!\!\!\!\!\!\!\!\!\!\!\!\!\!\!\!\!\!\!\!
		\parbox{13.5cm}{nếu $Z(x,x')$ thỏa mãn thì với mọi $r \in \SigmaDagOR$, tồn tại một song ánh \mbox{$h: \{y \in \Delta^\mI \mid r^\mI(x,y)\} \to \{y' \in \Delta^{\mI'} \mid r^{\mI'}(x',y')\}$} sao cho $h \subseteq Z$,} \label{bs:eqQ}
	\end{eqnarray}
	%
	nếu $\{\mQ,\mI\} \subseteq \Phi^\dag$ thì
	\begin{eqnarray}
	&&\!\!\!\!\!\!\!\!\!\!\!\!\!\!\!\!\!\!\!\!\!\!\!\!
		\parbox{13.5cm}{nếu $Z(x,x')$ thỏa mãn thì với mọi $r \in \SigmaDagOR$, tồn tại một song ánh \mbox{$h: \{y \in \Delta^\mI \mid r^\mI(y,x)\} \to \{y' \in \Delta^{\mI'} \mid r^{\mI'}(y',x')\}$} sao cho $h \subseteq Z$,} \label{bs:eqQI}
	\end{eqnarray}
	%
	nếu $\mU \in \Phi^\dag$ thì
	\begin{eqnarray}
		&&\!\!\!\!\!\!\!\!\!\!\!\!\!\!\!\!\!\!\!\!\!\!\!\! 
		\V x \in \Delta^\mI,\ \E x' \in \Delta^{\mI'}, Z(x,x') \label{bs:eqU1} \\
		&&\!\!\!\!\!\!\!\!\!\!\!\!\!\!\!\!\!\!\!\!\!\!\!\! 
		\V x' \in \Delta^{\mI'},\ \E x \in \Delta^\mI, Z(x,x'), \label{bs:eqU2}
	\end{eqnarray}
	%
	nếu $\Self \in \Phi^\dag$ thì
	\begin{eqnarray}
		&&\!\!\!\!\!\!\!\!\!\!\!\!\!\!\!\!\!\!\!\!\!
		Z(x,x') \Rightarrow [r^\mI(x,x) \Leftrightarrow r^{\mI'}(x',x')], \label{bs:eqSelf}
	\end{eqnarray}
	trong đó $\sharp\Gamma$ ký hiệu cho lực lượng của tập hợp $\Gamma$.\myend
\end{Definition}

\begin{Lemma}
\label{lm:Bisimulation}~
	\begin{enumerate}
		\item Quan hệ $\{\tuple{x,x} \mid x \in \Delta^\mI\}$ là một $\mLSPD$-mô phỏng hai chiều giữa $\mI$ và $\mI$.\label{lm:item1}
		%  
		\item Nếu $Z$ là một $\mLSPD$-mô phỏng hai chiều giữa $\mI$ và $\mI'$ thì $Z^{-1}$ cũng là một $\mLSPD$-mô phỏng hai chiều giữa $\mI'$ và $\mI$.\label{lm:item2}
		%  
		\item Nếu $Z_1$ là một $\mLSPD$-mô phỏng hai chiều giữa $\mI_0$ và $\mI_1$, $Z_2$ là một $\mLSPD$-mô phỏng hai chiều giữa $\mI_1$ và $\mI_2$ thì $Z_1 \circ Z_2$ là một $\mLSPD$-mô phỏng hai chiều giữa $\mI_0$ và~$\mI_2$.\label{lm:item3}
		%
		\item Nếu $\mathcal{Z}$ là một tập các $\mLSPD$-mô phỏng hai chiều giữa $\mI$ và $\mI'$ thì $\bigcup \mathcal{Z}$ là một $\mLSPD$-mô phỏng hai chiều giữa $\mI$ và $\mI'$.\label{lm:item4}\myend
	\end{enumerate}
\end{Lemma}

\subsection{Quan hệ tương tự hai chiều và quan hệ tương đương}
\label{sec:Chap2.Bisimilary}

\begin{Definition}
\label{def:InterpretationBisimilarity}
	Cho $\mI$ và $\mI'$ là các diễn dịch trong ngôn ngữ $\mLSP$. Ta nói rằng $\mI$~{\em $\mLSPD$-tương tự hai chiều} với $\mI'$ nếu tồn tại một $\mLSPD$-mô phỏng hai chiều giữa $\mI$~và $\mI'$.\myend
\end{Definition}

\begin{Definition}
\label{def:ElementBisimilarity}
	Cho $\mI$ và $\mI'$ là các diễn dịch trong ngôn ngữ $\mLSP$, $x \in \Delta^\mI$ và $x' \in \Delta^{\mI'}$. Ta nói rằng $x$ {\em $\mLSPD$-tương tự hai chiều} với $x'$ nếu tồn tại một $\mLSPD$-mô phỏng hai chiều giữa $\mI$ và $\mI'$ sao cho $Z(x,x')$ thỏa mãn.\myend	
\end{Definition}

\begin{Definition}
\label{def:LSPEquivalence}
	Cho $\mI$ và $\mI'$ là các diễn dịch trong ngôn ngữ $\mLSP$, $x \in \Delta^\mI$ và $x' \in \Delta^{\mI'}$. Ta nói rằng $x$ {\em $\mLSPD$-tương đương} với $x'$ nếu với mọi khái niệm $C$ của $\mLSPD$, $x \in C^\mI$ khi và chỉ khi $x' \in C^{\mI'}$.\myend
\end{Definition}

Theo Bổ đề~\ref{lm:Bisimulation}, chúng ta thấy rằng quan hệ tương tự hai chiều giữa các diễn dịch là một quan hệ tương đương và quan hệ tương tự hai chiều giữa các phần tử trong diễn dịch cũng là một quan hệ tương đương.

%-------------------------------------------------------------------
\section{Tính bất biến đối với mô phỏng hai chiều}
\label{sec:Chap2.Invariant}
\subsection{Quan hệ giữa mô phỏng hai chiều với các khái niệm và vai trò}
\label{sec:Chap2.BisConceptRole}
\begin{Lemma}
	\label{lm:Condition}
	Cho $\mI$ và $\mI'$ là các diễn dịch trong ngôn ngữ $\mLSP$, $Z$ là một $\mLSPD$-mô phỏng hai chiều giữa $\mI$ và $\mI'$. Lúc đó, với mọi khái niệm $C$ của $\mLSPD$, mọi vai trò đối tượng $R$ của $\mLSPD$, mọi đối tượng $x, y \in \Delta^\mI$, $x', y' \in \Delta^{\mI'}$ và mọi cá thể $a \in \SigmaDagI$, các điều kiện sau sẽ được thỏa mãn:
	\begin{eqnarray}
	&&\!\!\!\!\!\!\!\!\!\!\!\!\!\!\!\!\!\!\!\!\! Z(x, x') \Rightarrow [C^\mI(x) \Leftrightarrow C^{\mI'}(x')] \label{bs:eqC3}\\
	&&\!\!\!\!\!\!\!\!\!\!\!\!\!\!\!\!\!\!\!\!\! [Z(x, x') \wedge R^\mI(x, y)] \Rightarrow \E y' \in \Delta^{\mI'} \mid [Z(y,y') \wedge R^{\mI'}(x',y')] \label{bs:eqR1}\\
	&&\!\!\!\!\!\!\!\!\!\!\!\!\!\!\!\!\!\!\!\!\! [Z(x, x') \wedge R^{\mI'}(x', y')] \Rightarrow \E y \in \Delta^\mI \mid [Z(y,y') \wedge R^\mI(x,y)], \label{bs:eqR2}
	\end{eqnarray}
	nếu $\mO \in \PhiDag$ thì:
	\begin{eqnarray}
	&&\!\!\!\!\!\!\! Z(x, x') \Rightarrow [R^\mI(x, a^\mI) \Leftrightarrow R^{\mI'}(x', a^{\mI'})].\quad\label{bs:eqOR}\ \myend
	\end{eqnarray}
\end{Lemma}

\subsection{Tính bất biến của khái niệm}
\label{sec:Chap2.ConceptInvariant}

\begin{Definition}[Khái niệm bất biến]
\label{def:InvariantConcept}
	Một khái niệm $C$ được gọi là {\em bất biến đối với $\mLSPD$-mô phỏng hai chiều} nếu $Z(x, x')$ thỏa mãn thì $x \in C^\mI$ khi và chỉ khi $x' \in C^{\mI'}$ với mọi diễn dịch $\mI$, $\mI'$ trong ngôn ngữ $\mLSP$ thỏa $\SigmaDag \subseteq \Sigma$, $\PhiDag \subseteq \Sigma$ và với mọi $\mLSPD$-mô phỏng hai chiều $Z$ giữa $\mI$ và~$\mI'$.\myend
\end{Definition}

\begin{Theorem}
	\label{th:ConceptInvariant}
	Tất cả các khái niệm của ngôn ngữ $\mLSPD$ đều bất biến đối với $\mLSPD$-mô phỏng hai chiều.\myend
\end{Theorem}

Định lý này cho phép mô hình hóa tính không phân biệt được của các đối tượng thông qua ngôn ngữ con $\mLSPD$. Tính không phân biệt của các đối tượng là một trong những đặc trưng cơ bản trong quá trình phân lớp dữ liệu. Điều này có nghĩa là chúng ta có thể sử dụng ngôn ngữ con $\mLSPD$ cho các bài toán học máy trong logic mô tả.

\subsection{Tính bất biến của cơ sở tri thức}
\label{sec:Chap2.KnowlwdgeBaseInvariant}

\begin{Definition}
\label{def:BoxInvariant}
	Một TBox $\mT$ (tương ứng, ABox $\mA$) trong $\mLSPD$ được gọi là {\em bất biến đối với $\mLSPD$-mô phỏng hai chiều} nếu với mọi diễn dịch $\mI$ và $\mI'$ trong $\mLSP$ tồn tại một $\mLSPD$-mô phỏng hai chiều giữa $\mI$ và $\mI'$ sao cho $\mI$ là mô hình của $\mT$ (tương ứng, $\mA$) khi và chỉ khi $\mI'$ là mô hình của $\mT$ (tương ứng, $\mA$).\myend
\end{Definition}

\begin{Corollary}
	\label{co:TBoxInvariant}
	Nếu $\mU \in \PhiDag$ thì tất cả các TBox trong $\mLSPD$ đều bất biến đối với $\mLSPD$-mô phỏng hai chiều.\myend
\end{Corollary}

Một diễn dịch $\mI$ trong $\mLSP$ được gọi là {\em kết nối đối tượng được đối với $\mLSPD$} nếu với mọi đối tượng $x \in \Delta^\mI$ tồn tại cá thể $a \in \SigmaDagI$, các đối tượng $x_0, x_1, \ldots, x_k \in \Delta^\mI$ và các vai trò đối tượng cơ bản $R_1, R_2, \ldots, R_k$ của $\mLSPD$ với $k \geq 0$ sao cho $x_0 = a^\mI$, $x_k = x$ và $R_i^{\mI}(x_{i-1}, x_i)$ thỏa mãn với mọi $1 \leq i \leq k$.

\begin{Theorem}
\label{th:TBoxInvariant}
	Cho $\mT$ là một TBox trong $\mLSPD$, $\mI$ và $\mI'$ là các diễn dịch trong $\mLSP$ thỏa điều kiện kết nối đối tượng được đối với $\mLSPD$ sao cho tồn tại một $\mLSPD$-mô phỏng hai chiều giữa $\mI$ và $\mI'$. Lúc đó $\mI$ là mô hình của $\mT$ khi và chỉ khi $\mI'$ là mô hình của $\mT$.\myend
\end{Theorem}

\begin{Theorem}
\label{th:ABoxInvariant}
	Cho $\mA$ là một ABox trong $\mLSPD$. Nếu $\mO \in \PhiDag$ hoặc $\mA$ chỉ chứa các khẳng định dạng $C(a)$ thì $\mA$ bất biến đối với $\mLSPD$-mô phỏng hai chiều.\myend
\end{Theorem}

\begin{Corollary}
\label{co:KnowledgeBaseInvariant}
	Cho cơ sở tri thức $\KB = \tuple{\mR, \mT, \mA}$ trong $\mLSPD$ sao cho $\mR = \emptyset$ và giả thiết $\mO \in \PhiDag$ hoặc $\mA$ chỉ chứa các khẳng định có dạng $C(a)$, $\mI$ và $\mI'$ là các diễn dịch kết nối đối tượng được trong $\mLSPD$ sao cho tồn tại một $\mLSPD$-mô phỏng hai chiều giữa $\mI$ và $\mI'$. Lúc đó $\mI$ là mô hình của $\KB$ khi và chỉ khi $\mI'$ là mô hình của $\KB$.\myend
\end{Corollary}

\section{Tính chất Hennessy-Milner đối với mô phỏng hai chiều}

\begin{Definition}
\label{def:FiniteImage}
	Một diễn dịch $\mI$ trong $\mLSP$ được gọi là {\em phân nhánh hữu hạn} (hay {\em hữu hạn ảnh}) đối với $\mLSPD$ nếu với mọi $x \in \Delta^\mI$ và với mọi vai trò $r \in \SigmaDagOR$ thì:
	\begin{itemize}
		\item tập $\{y \in \Delta^\mI \mid r^\mI(x,y)\}$ là hữu hạn,
		
		\item nếu $\mI \in \PhiDag$ thì tập $\{y \in \Delta^\mI \mid r^\mI(y, x)\}$ là hữu hạn.\myend
	\end{itemize}
\end{Definition}

\begin{Theorem}[Tính chất Hennessy-Milner]
\label{th:HennessyMilnerProperty}
	Cho $\Sigma$ và $\SigmaDag$ là các bộ ký tự logic mô tả sao cho $\SigmaDag \subseteq \Sigma$, $\Phi$ và $\PhiDag$ là tập các đặc trưng của logic mô tả sao cho $\PhiDag \subseteq \Phi$, $\mI$ và $\mI'$ là các diễn dịch trong $\mLSP$ thỏa mãn điều kiện phân nhánh hữu hạn đối với $\mLSPD$, sao cho với mọi $a \in \SigmaDagI$, $a^\mI$ $\mLSPD$-tương đương với $a^{\mI'}$. Giả thiết rằng $\mU \not \in \PhiDag$ hoặc $\SigmaDagI \not= \emptyset$. Lúc đó, $x \in \Delta^\mI$ $\mLSPD$-tương đương với $x' \in \Delta^{\mI'}$ khi và chỉ khi tồn tại một $\mLSPD$-mô phỏng hai chiều $Z$ giữa $\mI$ và $\mI'$ sao cho $Z(x, x')$ thỏa mãn.\myend
\end{Theorem}

\begin{Corollary}
\label{co:EquivalenceRelation}
	Cho $\Sigma$ và $\SigmaDag$ là các bộ ký tự logic mô tả sao cho $\SigmaDag \subseteq \Sigma$, $\Phi$ và $\PhiDag$ là tập các đặc trưng của logic mô tả sao cho $\PhiDag \subseteq \Phi$, $\mI$ và $\mI'$ là các diễn dịch trong $\mLSP$ thỏa điều kiện phân nhánh hữu hạn đối với $\mLSPD$. Giả thiết rằng $\SigmaDagI \not= \emptyset$ và với mọi $a \in \SigmaDagI$, $a^\mI$ $\mLSPD$-tương đương với $a^{\mI'}$.
	Lúc đó, quan hệ $\{\tuple{x, x'} \in \Delta^\mI \times \Delta^{\mI'} \mid x $ $\mLSPD$-tương đương với $x'\}$ là một $\mLSPD$-mô phỏng hai chiều giữa $\mI$ và $\mI'$.\myend
\end{Corollary}

\section{Tự mô phỏng hai chiều}
\label{sec:Chap2.AutoBisimulation}

\begin{Definition}[Tự mô phỏng hai chiều]
\label{def:AutoBisimulation}
	Cho $\mI$ là một diễn dịch trong $\mLSP$. Một {\em $\mLSPD$-tự mô phỏng hai chiều} của $\mI$ là một $\mLSPD$-mô phỏng hai chiều giữa $\mI$ và chính nó. Một $\mLSPD$-tự mô phỏng hai chiều $Z$ của $\mI$ được gọi là {\em lớn nhất} nếu với mọi $\mLSPD$-tự mô phỏng hai chiều $Z'$ của $\mI$ thì $Z' \subseteq Z$.\myend
\end{Definition}

Cho $\mI$ là một diễn dịch trong $\mLSP$, chúng ta ký hiệu $\mLSPD$-tự mô phỏng hai chiều lớn nhất của~$\mI$ là $\sim_\SdPdI$, và ký hiệu quan hệ nhị phân $\equiv_\SdPdI$ trên $\Delta^\mI$ là quan hệ thỏa mãn tính chất $x \equiv_\SdPdI x'$ khi và chỉ khi $x$ $\mLSPD$-tương đương với~$x'$.

\begin{Theorem} 
\label{th:AutoBisimulation}
	Cho $\Sigma$ và $\SigmaDag$ là các bộ ký tự của logic mô tả sao cho $\SigmaDag \subseteq \Sigma$, $\Phi$ và $\Phi^\dag$ là tập các đặc trưng của logic mô tả sao cho $\Phi^\dag \subseteq \Phi$, $\mI$ là một diễn dịch trong $\mLSP$. Lúc đó:
	\begin{enumerate}
		\item $\mLSPD$-tự mô phỏng hai chiều lớn nhất của $\mI$ tồn tại và nó là một quan hệ tương~đương,\label{th:AutoBisimulation-item1}  
		\item nếu $\mI$ là một phân nhánh hữu hạn đối với $\mLSPD$ thì quan hệ $\equiv_\SdPdI$ là một $\mLSPD$-tự mô phỏng hai chiều lớn nhất của $\mI$ (nghĩa là, quan hệ $\equiv_\SdPdI$ và $\sim_\SdPdI$ trùng khớp~nhau).\label{th:AutoBisimulation-item2}\myend
	\end{enumerate}
\end{Theorem}

Chúng ta nói rằng tập $Y$ {\em bị phân chia} bởi tập $X$ nếu $Y \setminus X \neq \emptyset$ và $Y \cap X \neq \emptyset$. Như vậy, tập $Y$ không bị phân chia bởi tập $X$ nếu hoặc $Y \subseteq X$ hoặc $Y \cap X = \emptyset$.
Phân hoạch $\mbY = \{Y_1, Y_2, \ldots,Y_n\}$ được gọi là {\em nhất quán} với tập $X$ nếu với mọi $1 \leq i \leq n$, $Y_i$ không bị phân chia bởi~$X$.

\begin{Theorem}
\label{th:Consistent}
	Cho $\Sigma$ và $\SigmaDag$ là các bộ ký tự của logic mô tả sao cho $\SigmaDag \subseteq \Sigma$, $\Phi$ và $\Phi^\dag$ là tập các đặc trưng của logic mô tả sao cho $\Phi^\dag \subseteq \Phi$, $\mI$ là một diễn dịch hữu hạn trong $\mLSP$ và $X \subseteq \Delta^\mI$. Gọi $\mbY$ là phân hoạch của $\Delta^\mI$ thông qua quan hệ $\sim_\SdPdI$. Lúc đó:
	\begin{enumerate}
		\item nếu tồn tại khái niệm $C$ của $\mLSPD$ sao cho $C^\mI = X$ thì phân hoạch $\mbY$ nhất quán với tập $X$,  
		\item nếu phân hoạch $\mbY$ nhất quán với tập $X$ thì tồn tại khái niệm $C$ của $\mLSPD$ sao cho $C^\mI = X$.\myend
	\end{enumerate}
\end{Theorem}

%%-------------------------------------------------------------------
\section*{Tiểu kết Chương~\ref{Chapter2}}
\addcontentsline{toc}{section}{Tiểu kết Chương~\ref{Chapter2}}
\label{sec:Chap2.Summary}
Thông qua ngôn ngữ $\mLSP$ và ngôn ngữ con $\mLSPD$, chương này đã trình bày mô phỏng hai chiều và tính bất biến đối với mô phỏng hai chiều trên một lớp các logic mô tả như đã đề cập trong Chương~\ref{Chapter1}. Các khái niệm, định nghĩa và các định lý, bổ đề cũng như các hệ quả được phát triển dựa trên các kết quả của các công trình~\cite{Divroodi2011B,Nguyen2013} với lớp các logic mô tả lớn hơn. Chúng tôi cũng trình bày các chứng minh cho những định lý, bổ đề, hệ quả đã nêu ra trong chương này. Tính bất biến, đặc biệt là tính bất biến của khái niệm là một trong những nền tảng cho phép mô hình hóa tính không phân biệt được của các đối tượng thông qua ngôn ngữ con. Tính không phân biệt của các đối tượng là một trong những đặc trưng cơ bản trong quá trình xây dựng các kỹ thuật phân lớp dữ liệu. Điều này có nghĩa là chúng ta có thể sử dụng ngôn ngữ con cho các bài toán học máy trong logic mô tả bằng cách sử dụng mô phỏng hai chiều.
\cleardoublepage